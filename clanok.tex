
\documentclass[10pt,twoside,slovak,a4paper]{article}

\usepackage[slovak]{babel}

\usepackage[IL2]{fontenc} 
\usepackage[utf8]{inputenc}
\usepackage{graphicx}
\usepackage{url}
\usepackage{hyperref} 
\usepackage{cite}
\usepackage{float}




\title{Technológia motion capture\thanks{Semestrálny projekt v predmete Metódy inžinierskej práce, ak. rok 2021/22, vedenie: Ing. Fedor Lehocki, PhD.}} 

\author{Patrik Krajčík\\[2pt]
	{\small Slovenská technická univerzita v Bratislave}\\
	{\small Fakulta informatiky a informačných technológií}\\
	{\small \texttt{xkrajcikp@stuba.sk}}
	}

\date{\small 24. október 2021} 



\begin{document}

\maketitle

\begin{abstract}
V mojom článku poskytujem informácie a všeobecný prehľad o motion capture technológií (ďalej už len mo-cap). Ďalej sa v mojom článku zameriavam na princípy fungovania tejto technológie a na systémy, ktoré využíva, pričom rozoberám výhody a nevýhody každej z nich. V krátkosti je zhrnutá história mo-cap technológie, od toho čo predchádzalo jej vzniku, kde, kedy a ako sa technológia prvýkrat využívala až po to, v akej podobe ju poznáme v dnešnej dobe. Hlavný dôvod prečo som si vybral práve túto problematiku je ten, že mám možnosť sa s ňou každodenne stretávať, či už vo filmoch, seriáloch alebo v rôznych počítačových hrách. Okrem zábavného priemyslu sa mo-cap technológia využíva aj v ďalších zaujímavých odvetiach akými je napríklad medicína, šport a ozbrojené sily. 
\end{abstract}


\section{Úvod}
Mo-cap technológia sa stala neoddeliteľnou súčasťou našich bežných životov. Bežne sa s ňou stretávame bez toho, aby sme si to vôbec uvedomili. To čo my dnes považujeme za samozrejmosť, napríklad animované 3D postavy, ktoré sa objavujú vo filmoch, bolo ešte v minulom storočí nemožné docieliť.  Na to, aby sa mohla mo-cap technológia používať takým spôsobom akým ju poznáme dnes, musela si prejsť pomerne dlhým štádiom vývoja.

Tvorba animácií zaznamenala výrazný pokrok až s príchodom počítačov. Dovtedy však bolo všetko potrebné robiť ručne, čo malo za následok, že tvorba animovaného filmu mohla trvať veľmi dlhú dobu. Za prvého predchodcu motion capture sa považuje rotoskopia. Väčšina starších animovaných filmov bola postavená práve na nej. Išlo o kreslenie obrázkov na filmový pás, pričom cieľom bolo vytvoriť ilúziu pohybu. Neskôr túto činnosť nahradili počítače, čo výrazne urýchlilo proces tvorby animácie. 

Prvýkrát sa  motion capture technológia objavila začiatkom 80. rokov v biomechanických laboratóriách. Až neskôr sa začal mo-cap využívať vo veľkom aj vo filmoch a videohrách. S prvou populárnou 3D animáciou vytvorenou práve vďaka mo-cap technológií sme sa mohli stretnúť vo filme Batman navždy. 
\end{document}
